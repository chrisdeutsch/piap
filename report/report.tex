\documentclass[11pt, a4paper]{article}

% Encoding
\usepackage[utf8]{inputenc}

% Hyphenation
\usepackage[english]{babel}

% Extended math environments
\usepackage{amsmath}
\numberwithin{equation}{section}

% Additional math fonts and symbols
\usepackage{amsfonts}
\usepackage{amssymb}

% Units and uncertainties
\usepackage{siunitx}
\sisetup{
    separate-uncertainty
}

% Margins
\usepackage[left=3.5cm, right=3.5cm, top=3cm, bottom=3cm, twoside]{geometry}

% Pictures
\usepackage{graphicx}

% Text color
\usepackage{color}

% Hyperlinks
\usepackage{hyperref}
\hypersetup{
    colorlinks = true,
    allcolors = {black}
}

% Better table layouts
\usepackage{booktabs}
\usepackage{multirow}
\usepackage{multicol}

% Page Header
\usepackage{fancyhdr}

% Float Barriers
\usepackage{placeins}

% Rotated figures
\usepackage{caption}
\usepackage{subcaption}
\usepackage{rotating}

% Wrapped figures
\usepackage{wrapfig}

\usepackage{float}

% Remarks
\newcommand{\remark}[1]{{\color{red}(#1)}}

% Code
\usepackage{listings}
\definecolor{dkgreen}{rgb}{0,0.6,0}
\definecolor{gray}{rgb}{0.5,0.5,0.5}
\definecolor{mauve}{rgb}{0.58,0,0.82}

\lstset{frame=tb,
	aboveskip=3mm,
	belowskip=3mm,
	showstringspaces=false,
	columns=flexible,
	basicstyle={\footnotesize\ttfamily},
	numbers=left,
	numbersep=4pt,
	numberstyle=\tiny\color{black},
	keywordstyle=\color{blue},
	commentstyle=\color{dkgreen},
	stringstyle=\color{mauve},
	breaklines=true,
	breakatwhitespace=true,
	tabsize=4,
	xleftmargin=1em,
	xrightmargin=0.8em
}

% Caption-Setup
\captionsetup{font={small}}
\renewcommand{\thefigure}{\thesection.\arabic{figure}}
\renewcommand{\thesubfigure}{\alph{subfigure}}
\renewcommand{\thetable}{\thesection.\arabic{table}}
\renewcommand{\thesubtable}{\alph{subtable}}

% Depth of TOC (Level: 1 sections, 2 subsections, 3 subsubsections)
\setcounter{tocdepth}{3}

% FANCYHDR SETUP
\pagestyle{fancy}
\fancyhead[EL,OR]{\thepage}
\fancyhead[ER]{\leftmark}
\fancyhead[OL]{\rightmark}

\renewcommand{\sectionmark}[1]{
\markboth{\thesection{} #1}{\thesection{} #1}
}
\renewcommand{\subsectionmark}[1]{
\markright{\thesubsection{} #1}
}

% Document Info
\title{Particles in a potential}

\author{Christopher Deutsch\footnote{christopher.deutsch@uni-bonn.de} \and Philip Hauer\footnote{philiphauer@googlemail.com}}

\date{\today}

\begin{document}

\begin{titlepage}

\maketitle

% ABSTRACT
\begin{abstract}
\noindent
This report summarizes the results obtained from simulating the canonical ensemble using a Random-Walk Metropolis algorithm.
A set of particles interacting via two different inter-particle potentials, the Coulomb potential with a hard core and the Lennard-Jones potential, is considered.
Moreover the average pair-distance of the particles in the ensemble is monitored with respect to the occurrence of phase transitions at specific temperatures.
Additionally the behavior of the system is analyzed for the two and three-dimensional case.
\end{abstract}

\end{titlepage}

% TABLE OF CONTENTS
\tableofcontents
% New page after TOC
\newpage

% CONTENT

\section{Introduction}
This report shall be concerned with the simulation and analysis of the canonical ensemble of particles interacting via different inter-particle potentials.
The potentials considered in the analysis are the Coulomb potential of charged particles with hard cores and the Lennard-Jones potential of neutral particles.
For the simulation of the ensemble the Random-Walk Metropolis algorithm is used, which is a Marko-Chain Monte-Carlo method and will be introduced in section \ref{sec:Metropolis}.

The emphasis of this report lies in the analysis of the temperature dependence of the canonical ensemble, with respect to phase transitions occurring at specific temperatures.
As an indicator of the phase of the ensemble an observable, the average pair-distance, is monitored over a range of temperatures.
Moreover the behavior of the ensemble in two and three-dimensions is analyzed and some visualizations of states explaining the different phases are created.

\section{Background}

\subsection{Canonical ensemble} \label{sec:Canonical_Ensemble}
\begin{wrapfigure}[14]{o}[20pt]{0.45\textwidth}
	\centering
	\includegraphics[width=0.4\textwidth]{./figures/canonical_ensemble.pdf}
	\caption{Schematic representation of the canonical ensemble.}
	\label{fig:canonical_ensemble}
\end{wrapfigure}
The statistical ensemble describing a fixed number of particles~$N$ in a given volume~$V$, which is in thermal equilibrium with a heat bath of temperature~$T$, is the canonical ensemble (cf.\ figure \ref{fig:canonical_ensemble}).
Aim of this project is to sample states from a given ensemble with fixed $N, V, T$ and use these to calculate expectation values of certain observables.
For this the probability density for realizing a specific microstate of the ensemble needs to be known.
In statistical mechanics the probability density for a canonical ensemble to be in a microstate with energy eigenvalue~$E$ is given by \cite{schwabl}
\begin{align*}
	P = \frac{1}{Z} \exp\left[ -\beta E \right] \qquad \beta := \frac{1}{k_\mathrm{B} T}
\end{align*}
with the normalization factor $1/Z$, where $Z$ is the canonical partition function.
For a system of particles with an inter-particle potential~$V_{ij}$ and neglecting kinetic energies the canonical partition function
\begin{align*}
	Z = \int \prod_{i=1}^N \mathrm{d}\mathbf{r}_i \exp\left[ -\beta \sum_{i < j} V_{ij} \right]
\end{align*}
is a constant of the ensemble and the total energy of the system is
\begin{align*}
	E = \sum_{i < j} V_{ij} \, \text{.}
\end{align*}
Since the calculation of $Z$ is computationally expensive a sampling method without the need of a normalized probability density is desirable.
One such method is sampling using the Metropolis algorithm, which is introduced in the following section.


\subsection{Metropolis algorithm} \label{sec:Metropolis}
The Metropolis algorithm is a Markov-chain Monte-Carlo method for generating an autocorrelated sequence of random samples of a given probability distribution.
These samples can be used to approximate distributions for which direct sampling methods cannot be employed.
Hereafter a summary of the algorithm is given.\\

\noindent\textbf{Random-Walk Metropolis algorithm:}\\
For a given initial state~$x_0$ and target density~$P(x)$, a sequence of random samples~$\left( x_n \right)$ can be obtained using the following steps:
\begin{enumerate}
	\item \textbf{Trial change:}
		Propose a new state $y$ according to a symmetric proposal distribution $q(Y|x_n)$ depending on the current state $x_n$.
		
	\item \textbf{Accept-reject step:}
		Calculate the acceptance probability
		\begin{align}
			\alpha(x, y) = \min\left\{ \frac{P(y)}{P(x_n)}, 1\right\} \,\text{.}
			\label{eq:acceptance_probability}
		\end{align}
		Generate $u \sim \mathcal{U}(0, 1)$ and set the random sample~$x_{n+1}$ according to:
		\begin{align*}
			x_{n+1} = \begin{cases}
					y \,, & u \leq \alpha(x_n, y_n) \\
					x_n \,, & u > \alpha(x_n, y_n)
				\end{cases}
		\end{align*}
		
	\item \textbf{Repeat}
\end{enumerate}
Moreover the target density~$P(x)$ does not need to be normalized, since it only occurs as a fraction of densities in the accept-reject step.
Therefore this algorithm is suitable to sample from a canonical ensemble without the explicit calculation of the canonical partition function.

Finally one has to consider the influence of the proposal scheme on the behavior of the algorithm.
A characteristic indicator of this is the mean acceptance rate in the accept-reject step.
Generally, a small acceptance rate leads to slow mixing of the Markov-Chain, which means that the state moves slowly through state-space.
Conversely a high acceptance gives rise to rapid mixing of the chain and a thorough exploration of the state-space.
In addition to this, the algorithm is less likely to get stuck at a local mode of the target distribution.
In practice one has to balance both effects to achieve optimal performance of the algorithm.
It is suggested \cite{gilks_richardson}, that a mean acceptance rate in the interval~$[0.15, 0.5]$ leads to good performance.

\section{Implementation} \label{sec:implementation}
For the simulations of the canonical ensemble programs written in the programming language \texttt{C++} were used.
In the following a summary of the core implementation details is given.

\subsection{Metropolis algorithm} \label{sec:Metropolis_alg}
Since the subject of this project requires the simulation of particles in two or three dimensions and different inter-particle potentials, a generic approach of implementing the algorithmic core was chosen.
The source code of the implementation is listed in appendix~\ref{sec:canonical_ensemble_source}.

The Metropolis algorithm is encapsulated in a class-template~\texttt{CanonicalEnsemble} taking a parameter \texttt{ParticleState} indicating the type used to describe the state of one particle.
For the case of two-dimensional charged particles this could be a \texttt{struct} containing three floating point members describing the particle charge~$q$ and two coordinates~$x, y$.
Each object of type \texttt{CanonicalEnsemble} has several important (inaccessible) members associated with it:
\begin{itemize}
	\item \texttt{state}:
		A dynamically allocated array of type \texttt{ParticleState} fully describing the state of the ensemble (i.e.\ the charges and coordinates of all particles).
		The current state can be queried using the \texttt{get\_state} member-function.
	
	\item \texttt{beta}:
		The thermodynamic beta~$\beta = (k_\mathrm{B} T)^{-1}$ of the heat bath associated with the canonical ensemble.
	
	\item \texttt{potential\_func}:
		A function object taking two particles and returning their inter-particle potential.
	
	\item \texttt{proposal\_func}:
		A function object taking a particle and a random number generator, which returns a new particle according to a user-defined proposal distribution.
\end{itemize}
These members are assigned at construction-time of a \texttt{CanonicalEnsemble}-object and therefore allow for full customization of the considered problem.
The underlying Metropolis algorithm is implemented in the \texttt{step} member-function:
\lstinputlisting[language=C++, firstline=59, lastline=79]{../code/src/CanonicalEnsemble.h}
In the following a detailed description of the implemented algorithm shall be given.
\begin{enumerate}
	\item
		A random particle of the currently realized state is selected using a uniform distribution of indices.
		The contribution to the total potential energy~$V$ of the chosen particle~$V_j$ is calculated using a helper-function \texttt{potential} according to
		\begin{align*}
			V_j = \sum_{\substack{i = 1 \\ i \neq j}}^N V_{ij}
		\end{align*}
		with the inter-particle potential $V_{ij}$ as specified in \texttt{potential\_func}.
		
	\item
		The state of the selected particle is saved and a trial change of the current state is made using the proposal distribution as implemented in \texttt{proposal\_func} (further details in section \ref{sec:proposal_function}).
		Then the contribution of the proposed particle to the total potential energy~$V$ is calculated.
		
	\item
		At this point the target distribution of the canonical ensemble
		\begin{align*}
			P \propto \exp\left[ - \beta E \right]
		\end{align*}
		with the total energy of the system
		\begin{align*}
			E = \sum_{i < j} V_{ij}
		\end{align*}
		has to be considered.
		To obtain the acceptance probability~$\alpha$ according to equation \eqref{eq:acceptance_probability} the ratio
		\begin{align} \label{Eq:Prob_New_State}
			\alpha = \frac{\exp\left[ -\beta E_\mathrm{prop.} \right]}{\exp \left[ -\beta E_\mathrm{cur.} \right]} = \exp\left[ -\beta \left( E_\mathrm{prop.} - E_\mathrm{cur.} \right) \right]
		\end{align}
		is calculated.
		Since only one particle was changed, the difference $E_\mathrm{prop.} - E_\mathrm{cur.}$ reduces to the difference in contributions to the potential energy of the proposed and the current particle state.
		
		Lastly a $\mathcal{U}(0, 1)$-distributed random variable is compared to the calculated acceptance probability.
		If the variable is smaller than the acceptance probability the trial configuration is rejected, the old state is restored and the member function returns \texttt{false} indicating that the proposed state was rejected.
		Otherwise the new proposed state is kept and \texttt{true} is returned.		
\end{enumerate}

\subsection{Proposal function} \label{sec:proposal_function}
The proposal function contains an implementation of the proposal distribution, which occurs in the Metropolis algorithm.
Its required arguments are a reference to a \texttt{ParticleState} representing the state of one particle, for which a new state according to the proposal distribution should be chosen.
Moreover the function takes a reference to a random number generator, that is used to create random variables for sampling from the proposal distribution.

For illustration purposes consider the one dimensional case of a particle with charge~$q$ and a coordinate~$x$ on a line.
A proposal distribution
\begin{align*}
	q(y|x) = \begin{cases}
		\frac{1}{2\Delta} \,, & y \in \left[x - \Delta, x + \Delta \right]\\
		0 \,, & \text{else}
	\end{cases}
\end{align*}
suggesting a new particle position uniformly on a line segment of length $2\Delta$ centered at the current particle position~$x$.
This could be implemented as a proposal function in the following way:
\begin{lstlisting}[language=C++]
struct Particle1D {
	double q;
	double x;
}

Particle1D proposal_function_1d(const Particle1D &p, std::m19937 &rng) {
	std::uniform_real_distribution<> unif_dist(-delta, delta);
	
	Particle1D ret;
	ret.q = p.q;
	ret.x = p.x + unif_dist(rng);
	
	return ret;
}
\end{lstlisting}
(for this to be valid code the parameter \texttt{delta} must be set\footnote{For the implementation used in this report, proposal functions are realized as function objects with an associated member \texttt{delta}.}).
This scheme is easily extended to charged/uncharged particles in two or three dimensions, using an uniform proposal in a box centered at the particle position.
The simulated data used in this report was generated using such a proposal scheme.


\section{Temperature dependence of the canonical ensemble} \label{sec:Temperature}
To analyze the temperature dependence of the canonical ensemble using different inter-particle potentials, which will be introduced in section \ref{sec:Potentials}, the implementation as described in section~\ref{sec:implementation} is used.
At first some important settings of the simulation, which vary depending on the case considered, shall be introduced.

The initial state of the simulation is a uniform distribution of~$N$ charged or uncharged particles inside a two or three dimensional box with side length~$L$.
Since a random initial state is unprobable according to the target distribution, a burn-in phase discarding the first 1000 samples of the Markov-Chain will be employed.

For monitoring the phase transitions of the ensemble, the average pair-distance (introduced in section \ref{sec:Temp_Dep}) can be observed over a specific temperature range or more precisely the thermodynamic~$\beta$, which needs to be set beforehand.
As explained in section \ref{sec:Metropolis} the mean acceptance rate is critical for good performance of the algorithm, which strongly depends on the simulated temperature and the parameter~$\Delta$ of the proposal function.
Initial testing of the algorithm showed that reasonable performance can be obtained with an acceptance rate of~\SI{30}{\percent}.
Therefore the parameter~$\Delta$ of the proposal function has to be adapted depending on the considered temperature to achieve the desired acceptance rate.

Despite carefully setting the acceptance rate, the Random-Walk Metropolis algorithm often gets stuck at local modes of the target probability density.
Therefore an approach of simulating a number of shorter chains was chosen.
The chosen parameters of the simulation will be stated in the corresponding sections.

\subsection{Potentials} \label{sec:Potentials}
In the following a short introduction of the inter-particle potentials used in the simulation of the canonical ensemble is given.
For computational simplicity all quantities are considered to be dimensionless and charges are assumed to be $\pm 1$.

\begin{description}
	\item[Coulomb potential with hard core:]
		The Coulomb potential of two charged particles $V_\mathrm{Coulomb} \propto r^{-1}$ and a short ranged force $V_\mathrm{core} \propto r^{-8}$ parameterizing the particle's hard core  is given as
		\begin{align*}
			V_{ij} = \frac{q_i q_j}{| \mathbf{r}_i - \mathbf{r}_j |} + \frac{1}{| \mathbf{r}_i - \mathbf{r}_j |^8}
		\end{align*}
		with particle charges~$q$ and position vectors~$\mathbf{r}$.
		At intermediate ranges this force is attractive for oppositley charged particles and repulsive for particles of the same charge.
		The general behavior of this potential is depicted in figure \ref{fig:coulomb_core}, in which the hard core of the potential is visible as the steep ascend of the potential energy at a distance of $r \approx 1$.
		Moreover a prominent minimum of the potential for oppositley charged particles can be observed at $r \approx 1.35$.
		This is approximately the lattice spacing of crystals formed at low temperatures, as can be seen in section \ref{sec:2d_coulomb_vis}.
		
	\item[Lennard-Jones potential:]
		The Lennard-Jones potential describes the inter-atomic potential between a pair of neutral atoms interacting via attractive Van der Waals forces $V_\mathrm{VdW} \propto r^{-6}$ and a repulsive contribution due to the Pauli exclusion principle $V_\mathrm{Pauli} \propto r^{-12}$.
		Without loss of generality one can assume the functional form
		\begin{align*}
			V_{ij} = \frac{1}{| \mathbf{r}_i - \mathbf{r}_j |^{12}} - \frac{1}{| \mathbf{r}_i - \mathbf{r}_j |^6}
		\end{align*}
		with a minimum of the potential at $r\approx \num{1.12}$.
		This is depicted in figure \ref{fig:lennard_jones} and is similar to the Coulomb potential of oppositley charged particles.
		However the Lennard-Jones potential is extremely short ranged due to its $r^{-6}$-dependence as compared to the $r^{-1}$-dependence of the Coulomb potential.
				
\end{description}

\begin{figure}[p]
	\begin{subfigure}{1.0\textwidth}
		\centering
		\includegraphics{./figures/potential_coulomb.pdf}
		\subcaption{Coulomb potential with hard core for particles of same and opposite charge.}
		\label{fig:coulomb_core}
	\end{subfigure}
	\begin{subfigure}{1.0\textwidth}
		\centering
		\includegraphics{./figures/potential_lennard_jones.pdf}
		\subcaption{Lennard-Jones potential.}
		\label{fig:lennard_jones}
	\end{subfigure}
	\caption{Inter-particle potentials considered in the simulation of the canonical ensemble.}	
\end{figure}


\subsection{Analysis of the temperature dependence} \label{sec:Temp_Dep}
As already stated the temperature dependence of the canonical ensemble can be monitored using the average pair-distance~$d$ which is given by
\begin{align*}
d = \frac{2}{N(N - 1)} \sum_{i<j} |\mathbf{r}_i - \mathbf{r}_j | \, \text{,}
\end{align*}
where $N$ is the number of particles and $\mathbf{r}_i$, $\mathbf{r}_j$ are the positions of the $i$-th and $j$-th particle.
The calculation of the average pair-distance is only done every 20th step of the Markov-Chain in order to save computation time\footnote{A comparison between the calculation of the observable at every and every 20th step was done and no significant difference in the calculated expectation value was observed.}.
The expectation value of the average pair-distance can then be determined and saved for later analysis.
In order to estimate the uncertainty of the average pair-distance, the standard deviation and 0.05/0.95-quantiles of the expectation values from several chains at the same temperature are used.

In the following sections the average pair-distance~$d$ was simulated for different thermodynamic~$\beta$ and is depicted in $d(\beta)$-graphs using bands to indicate the $1\sigma$-range and quantiles.

\subsubsection{Coulomb potential with hard core in two dimensions} \label{sec:2d_coulomb_tempdep}
For this case a total of 20 positively and 20 negatively charged particles were simulated in a two dimensional box with side length 15.0.
The temperature range of this simulation was chosen to be $\beta \in [1, 500]$, which contains the relevant phase transitions of the ensemble.
Since this range consists of multiple orders of magnitude, the temperature is best depicted using a logarithmic axis.
Therefore the simulated thermodynamic~$\beta$ should increase exponentially to achieve equally spaced points on the logarithmic axis.
Explicitly the simulation starts at $\beta = 1.0$ and increases by approximately \SI{4}{\percent} for each step, which results in a total of 160 simulated temperatures on the range considered.
Moreover for every thermodynamic~$\beta$ a total of \num{400} chains (with $10^6$ samples each) were generated.

The resulting average pair-distance plot is given in figure \ref{Fig:Temp_dep_Cou2D}, where the red line gives the mean value of the average pair-distance over all chains at a fixed temperature and analogously the red band the range of one standard deviation at either side of the mean.

\begin{figure}[h]
	\centering
	\includegraphics{./figures/temp_dep_coulomb2d.pdf}
	\caption{Temperature dependence of the average pair-distance for a Coulomb potential with hard cores in two dimensions.
		A total of 20 positively and 20 negatively charged particles in a box with side length 15.0 were simulated.}
	\label{Fig:Temp_dep_Cou2D}
\end{figure}

Generally the diagram can be split into three distinct parts, which will be analyzed in the following paragraphs.
For a better understanding one can take a preemptive look at the figures in section \ref{sec:Visualisation}.
\begin{description}
	\item[Gaseous state $(\beta \lesssim 10^1)$] 
		The average pair-distance is large for high temperatures or equivalently small $\beta$.
		One expects this behavior, since for high temperatures the system is in a gaseous state and the particles are homogeneously distributed in the whole volume of the box.
		Therefore the average pair-distance is maximum in a gaseous state.
		Moreover one observes small uncertainties in this temperature range because of the homogeneous distribution of the particles.
	
	\item[Liquid state $(10^1 \lesssim \beta \lesssim 10^2)$]
		For higher $\beta$ the average pair-distance decreases.
		This happens at $\beta \approx 10$ at which the system enters a liquid-like state.
		The particles start to form chains, which lead to a reduction in the average pair-distance.
		
		This can be understood if one takes a closer look to equation \ref{Eq:Prob_New_State}.
		The probability that a new proposed state is accepted depends on the value of the thermodynamic~$\beta$.
		With higher $\beta$ the acceptance probability~$\alpha$ gets smaller if a trial state with higher energy is proposed.
		Therefore the Metropolis algorithm tends to minimize the total energy of the state for low temperatures.
		
		In the physical sense this leads to the formation of amorphous structures in a liquid-like state, which tend to reduce the total energy of the system at intermediate temperatures.		
		
	\item[Solid state $(\beta \gtrsim 10^2)$]
		At extremely low temperatures the average pair-distance is almost constant and minimum.
		This is interpreted as the particles forming a lattice and thus entering a solid (crystal-like) state.
		If a regular lattice is formed, the particles cannot get any closer for smaller temperatures, since the particles tend to coincide with the potential minimum as discussed in section \ref{sec:Potentials}.
		
		The reason (concerning the algorithm) that the particles form a solid state with a small average pair-distance is analogous to the argumentation for the liquid state.
		
		Finally one can observe a big uncertainty in figure \ref{Fig:Temp_dep_Cou2D} for the average pair-distance at high $\beta$.
		This is a result of the formation of crystallographic defects in the lattice.
		If the solid is in a state where it is not possible to form a compact grid (due to vacancies or appendices on the crystal), the resulting average pair-distance tends to increase.
		Therefore the formation of crystallographic defects leads to a large uncertainty in the average pair-distance.
\end{description}


\subsubsection{Coulomb potential with hard core in three dimensions}
Similarly to the previous section, the canonical ensemble is simulated using the Coulomb potential between particles with hard cores (still using 20 positively and 20 negatively charged particles).
However this time the particles are considered to be in a three-dimensional box instead of a two-dimensional plane.
To get a particle density that is comparable to the two-dimensional case, the side length of the box was reduced to \num{8.0}.
The temperature range is unchanged and each simulated chain still contains $10^6$ samples, but the number of chains at each fixed temperature could be reduced to \num{240} to obtain sufficient results.
The resulting average pair-distance is depicted in figure \ref{fig:temp_dep_cou3d}.
\begin{figure}[h]
	\centering
	\includegraphics{./figures/temp_dep_coulomb3d.pdf}
	\caption{Temperature dependence of the average pair-distance for a Coulomb potential with hard cores in three dimensions.
		A total of 20 positively and 20 negatively charged particles in a box with side length 8.0 were simulated.}
	\label{fig:temp_dep_cou3d}
\end{figure}

The general behavior of the mean of the average pair-distance is similar to the two-dimensional case.
Therefore the explanations of the phase transitions of the ensemble transfer to the three-dimensional case.
Nevertheless significant differences in the calculated standard deviations in the low temperature regimes can be observed.
These can be attributed to the additional degree of freedom in the particle's movement, which leads to a higher probability of resolving the crystallographic defects during the formation of a lattice.
Moreover the sphere packing density in three dimensions is bigger than in the two-dimensional case and thus lead to smaller variations in the calculated average pair-distance at low temperatures.


\subsubsection{Lennard-Jones potential in two dimensions}
\label{sec:lennard_jones_2d}
For the simulation of the Lennard-Jones potential in two dimensions the side length of the box was chosen to be \num{12.0} and a total of 40 neutral particles were considered.
In addition to this the simulated temperature range was shifted to $\beta \in [0.1, 100.0]$ to enclose the phase transitions of the ensemble.
On this range a total of 180 temperatures with 150 chains per step were simulated.
Each chain contains \num{200000} samples.
The resulting average pair-distance is depicted in figure \ref{Fig:Temp_dep_LJ2D}.
\begin{figure}[h]
	\centering
	\includegraphics{./figures/temp_dep_lennard_jones2d.pdf}
	\caption{Temperature dependence of the average pair-distance for the Lennard-Jones potential in two dimensions.
		A total of 40 neutral particles in a box with side length 12.0 were simulated.}
	\label{Fig:Temp_dep_LJ2D}
\end{figure}
As before the diagram can be split into three distinct parts:
\begin{description}
	\item[Gaseous state $(\beta \lesssim 2)$]
		For high temperatures the behavior of the average pair-distance mirrors the one for the case of a Coulomb inter-particle potential.
		In this temperature range the particles form a gaseous state with a homogeneous distribution of particles.
		This state is barely affected by the type of potential used in the simulation.
	
	\item[Liquid state $(2 \lesssim \beta \lesssim 7)$]
		For intermediate temperatures the particles start to form small clusters and chains in a liquid-like state and thus reducing the average pair-distance.
		A notable difference to the Coulomb case is, that the average pair-distance decreases more quickly because the Lennard-Jones force is always attractive at intermediate and long ranges.
		In contrast to this, the Coulomb force is always repulsive for oppositely charged particles which leads to a less rapid decrease of the average pair-distance during the phase transition from gas to liquid state.		
	
	\item[Solid state $(\beta \gtrsim 7)$]
		At a thermodynamic $\beta$ of approximately 8 the particles form a densely packed cluster and thus minimizing the average pair-distance in a solid-like state.
		A remarkable behavior sets in at even lower temperatures, where the average pair-distance starts to rise again (with large uncertainties).
		This is due to the formation of several smaller clusters in the simulated volume, which cannot merge due to the Lennard-Jones forced being extremely short-ranged.
		For higher temperatures in the solid state regime this does not occur, since the remaining thermal movement of the particles tends to merge smaller clusters into a single one and thus reducing the average pair-distance and its variance.	
\end{description}
Some visualizations of simulated states for different temperatures are given in the following section \ref{sec:Visualisation}.

\subsubsection{Lennard-Jones potential in three dimensions}
Lastly the three-dimensional case of the Lennard-Jones potential is considered.
Except for the side length of the simulated box, which was decreased to 5.0 to get approximately the same density of particles, all parameters of the simulation are identical to the two-dimensional case.
The corresponding average pair-distance plot is shown in figure \ref{Fig:Temp_dep_LJ3D}.
\begin{figure}
	\centering
	\includegraphics{./figures/temp_dep_lennard_jones3d.pdf}
	\caption{Temperature dependence of the average pair-distance for the Lennard-Jones potential in two dimensions.
		A total of 40 neutral particles in a box with side length 5.0 were simulated.}
	\label{Fig:Temp_dep_LJ3D}
\end{figure}

The phase transitions take place at approximately the same temperatures as for the two-dimensional case.
However the behavior at very small temperatures differs significantly, which can be traced back to the parameters of the simulation.
Since the box size was reduced to attain a similar particle density as in the two-dimensional case, the range of the Lennard-Jones force is sufficient to form a single cluster of particles at low temperatures.
For a larger volume the behavior in figure \ref{Fig:Temp_dep_LJ2D} is reproduced.

\subsection{Visualization of ensemble states} \label{sec:Visualisation}
In this section a few visualization of sampled states are depicted.
For all states depicted in this section a simulated volume with side length 15.0 was used.

\subsubsection{Coulomb potential with hard core in three dimensions} \label{sec:2d_coulomb_vis}
\begin{figure}[hp]
	\begin{subfigure}[t]{0.48\textwidth}
		\centering
		\includegraphics[width=\textwidth]{figures/Kristall_3_beta_500.pdf}
		\subcaption{Regular lattice for $\beta = 500$.}
		\label{fig:coulomb_crystal}
		\vspace*{0.3cm}
	\end{subfigure}
	\hfill
	\begin{subfigure}[t]{0.48\textwidth}
		\centering
		\includegraphics[width=\textwidth]{figures/Kristall_4_beta_500.pdf}
		\subcaption{Lattice with defects for $\beta = 500$.}
		\label{fig:coulomb_crystal_defects}
		\vspace*{0.3cm}
	\end{subfigure}
	\begin{subfigure}[t]{0.48\textwidth}
		\centering
		\includegraphics[width=\textwidth]{figures/Fluid_1_beta_40.pdf}
		\subcaption{Amorphous state for $\beta = 40$.}
		\label{fig:coulomb_amorphous}
		\vspace*{0.3cm}
	\end{subfigure}
	\hfill
	\begin{subfigure}[t]{0.48\textwidth}
		\centering
		\includegraphics[width=\textwidth]{figures/Gas_1_beta_10.pdf}
		\subcaption{Molecular gas for $\beta = 10$.}
		\label{fig:coulomb_molecular}
		\vspace*{0.3cm}
	\end{subfigure}
	\centering
	\begin{subfigure}[t]{0.48\textwidth}
		\centering
		\includegraphics[width=\textwidth]{figures/Plasma_1_beta_2.pdf}
		\subcaption{Single particle gas (ionized) for $\beta = 2$.}
		\label{fig:coulomb_ion}
		\vspace*{0.3cm}
	\end{subfigure}
	\caption{Excerpt of simulated states of the canonical ensemble using the Coulomb potential with hard particle cores.}
	\label{fig:coulomb_vis}
\end{figure}
In figure \ref{fig:coulomb_vis} some states of the canonical ensemble are visualized.
In the following a short description of the depicted states is given.
\begin{description}
	\item[Figure \ref{fig:coulomb_crystal}]
		Particles forming a crystal (solid state) without major defects at very low temperatures.
		The average pair-distance is small for such states.
		The lattice forms a quadrangular grid due to the repulsive contribution of particles of the same charge.
		This is different for the Lennard-Jones potential.
		Except for a small deformation, no crystallographic defects occur.
	
	\item[Figure \ref{fig:coulomb_crystal_defects}]
		A solid state with a irregular lattice.
		The average pair-distance is larger than in the case of a regular lattice.
		These defects are a source of the large uncertainties of the average pair-distance as discussed in section \ref{sec:2d_coulomb_tempdep}.
	
	\item[Figure \ref{fig:coulomb_amorphous}]
		An amorphous state (fluid) at intermediate temperatures.
		The average pair-distance is smaller than for the gaseous case, because of small clusters and chains formed by the particles.
		
	\item[Figure \ref{fig:coulomb_molecular}] 
		A state at higher temperatures forming a molecular gas (freely moving pairs of positive and negatively charged particles).
		The average pair-distance is at its maximum, since the pairs are distributed homogeneously in the entire volume.
	
	\item[Figure \ref{fig:coulomb_ion}]
		At very high temperatures the particle pairs break apart and the gas enters a plasma-like state with charged particles moving independently in the volume.
		
\end{description}

\subsubsection{Two-dimensional case with Lennard-Jones potential}
In figure \ref{fig:lj_vis} some states of the canonical ensemble are visualized.
In the following a short description of the depicted states is given.
\begin{figure}[h]
	\begin{subfigure}[t]{0.48\textwidth}
		\centering
		\includegraphics[width=\textwidth]{figures/Beta_100_LJ.pdf}
		\subcaption{Densely packed clusters for $\beta = 100$.}
		\label{fig:lj_multiple_clusters}
		\vspace*{0.3cm}
	\end{subfigure}
	\hfill
	\begin{subfigure}[t]{0.48\textwidth}
		\centering
		\includegraphics[width=\textwidth]{figures/Beta_10_LJ.pdf}
		\subcaption{Single cluster for $\beta = 10$.}
		\label{fig:lj_single_cluster}
		\vspace*{0.3cm}
	\end{subfigure}
	\begin{subfigure}[t]{0.48\textwidth}
		\centering
		\includegraphics[width=\textwidth]{figures/Beta_5_LJ.pdf}
		\subcaption{Fluid-like state for $\beta = 5$.}
		\label{fig:lj_chain}
		\vspace*{0.3cm}
	\end{subfigure}
	\hfill
	\begin{subfigure}[t]{0.48\textwidth}
		\centering
		\includegraphics[width=\textwidth]{figures/Beta_1_LJ.pdf}
		\subcaption{Single particle gas for $\beta = 1$.}
		\label{fig:lj_gas}
		\vspace*{0.3cm}
	\end{subfigure}
	\caption{Excerpt of simulated states of the canonical ensemble using the Lennard-Jones potential.}
	\label{fig:lj_vis}
\end{figure}
\begin{description}
	\item[Figure \ref{fig:lj_multiple_clusters}] 
		Formation of densely packed clusters for small temperatures.
		As described in section \ref{sec:lennard_jones_2d} the clusters are independent due to the short range of the Lennard-Jones potential.
		As opposed to the Coulomb potential, the Lennard-Jones case does not contain any repulsive charges.
		Therefore then particles form a triangular grid.
		
	\item[Figure \ref{fig:lj_single_cluster}]
		At higher temperatures a single but more loosely packed cluster forms.
	
	\item[Figure \ref{fig:lj_chain}]
		A fluid-like state with particles forming chains and pairs.
		This is comparable to the amorphous state for the Coulomb potential.
	
	\item[Figure \ref{fig:lj_gas}] 
		A single particle gas identical to the ionized gas of the Coulomb case.
		In both cases no structures are visible and particles are evenly distributed in the volume.
\end{description}



\remark{3D Visualization?}

\clearpage
\section{Conclusion}

In summary, it can be stated that the project was a success.
The results can be explained by theory and the program has worked properly.
Nevertheless the project will be critically summarized.
\begin{description}
\item[Theory]
	The underlying theory known from statistical mechanics was not very demanding, since a probabilistic solution was employed.
	In contrast to that understanding theory behind the computational aspects of the problem was very interesting.
	Therefore this project is a good fit for learning more about computational physics, especially for sampling from multidimensional distributions.
	The suggested algorithm is a good instrument to analyze the various ensembles as they occur in statistical mechanics.

\item[Implementation]
	Once the concept of the Random-Walk Metropolis algorithm was understood, the realization in an efficient implementation was not difficult.
	Moreover a generic approach was employed to enable the simulation of arbitrary potentials.
	Therefore it was easy to extend the simulation to the three-dimensional case.

\item[Generation of Data]
	The $d(\beta)$-plots are very demanding in terms of the computation time needed to obtain smooth graphs.
	Therefore the data used for this report was generated over several days to gain reasonable results.
	Nevertheless the results show that the investment of a few days of computation time was worthwhile.

\item[Analysis of the Results]
	The analysis of the generated data was straightforward.
	Animating the states of the Metropolis algorithm gave a good insight into its behavior.
	Moreover the results could easily be connected to the physical background.
\end{description}
As a final conclusion this project gave great insight into the field of computational physics.
Moreover the exploration of different approaches to solving the problem at hand was very interesting.

\FloatBarrier
% BIBLIOGRAPHY
\vspace{\fill}
\begin{thebibliography}{9}
\bibitem{schwabl}
	F.\ Schwabl,
	\emph{Statistische Mechanik},
	Springer, 3rd edition 2006.

\bibitem{gilks_richardson}
	W.\ R.\ Gilks, S.\ Richardson, D.\ J.\ Spiegelhalter,
	\emph{Markov Chain Monte Carlo in Practice},
	Chapman and Hall/CRC, 1st edition 1995.
\end{thebibliography}

\begin{appendix}
	\newpage
	\section{Source code}
	The full source code can be found on \url{https://github.com/chrisieh/piap} (a \texttt{C++14} compliant compiler is needed).
	\subsection{CanonicalEnsemble{.}h} \label{sec:canonical_ensemble_source}
	\lstinputlisting[language=C++]{../code/src/CanonicalEnsemble.h}
\end{appendix}

\end{document}